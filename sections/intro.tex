\section{Introduzione}

	\subsection{Il Gioco}
			
			
		\subsection{Elementi di gioco}
		
		\subsection{Menu}
		
		\subsection{Sonoro}
	
	
	
\newpage
	
	\subsection{I Frameworks}
		Oggigiorno esistono numerosi framework per sviluppare videogame mobile multipiattaforma. \textbf{Unity3D} e \textbf{Corona SDK} sono due di questi e sono stati scelti per sperimentarne le differenze nel processo di sviluppo di Flappy Flame.
		
		\subsection{Unity3D}
		
		\subsection{Corona SDK}
			Corona SDK è un framework cross-platform per lo sviluppo di app, in particolare offre delle API per un motore grafico 2D, essenziale per minimizzare lo sviluppo di videogame. Il framework si appoggia al linguaggio di scripting Lua: linguaggio molto famoso nel mondo degli sviluppatori di videogame data la sua alta efficienza e facilità di utilizzo. FONTE??? Il framework può essere esteso grazie a Plugin e permette di interfacciarsi alle API native delle maggiori piattaforme qualora ci sia la necessità. Inoltre è completamente gratuito (è richiesta solo la registrazione) e i software da esso prodotti possono essere commercializzati senza nessun costo aggiuntivo.
		
		
	\subsection{Struttura del documento}
		La presente sezione è stata scritta da entrambi gli autori. Le sezioni che coinvolgono la discussione dello sviluppo con lo specifico framework sono scritte dall'autore segnalato sotto al titolo della sezione. Le due sezioni \ref{sec:unity} e \ref{sec:corona} sono indipendenti tra loro e data la notevole differenza dei due framework non seguono la stessa struttura. Nell'ultima sezione invece si metteranno a confronto il processo e i risultati ottenuti dall'utilizzo di questi framework.
	
	