
\newpage

\section{Comparazione Unity3D e Corona SDK}

	\subsection*{Installazione}
		\paragraph{Unity3D} Lunga, pesante e necessita di elementi aggiuntivi post-installazione in alcuni casi. \textbf{Voto 5}
		\paragraph{Corona SDK} Nessun problema, il tempo di registrarsi e tutto è pronto per iniziare a programmare. \textbf{Voto 9}
		
	\subsection*{Strumenti di sviluppo}
		\paragraph{Unity3D} Molto completi e personalizzabili, inizialmente complessi, ma con poco diventano intuitivi. \textbf{Voto 8.5}
		\paragraph{Corona SDK} Leggero e performante l'emulatore offerto. Manca un IDE che integri il tutto. \textbf{Voto 6.5}
	
	\subsection*{Documentazione}
		\paragraph{Unity3D} Completa in tutti i suoi campi. \textbf{Voto 10}
		
		\paragraph{Corona SDK} Ben documentato per iniziare. Mancano approfondimenti necessari per usare tutte le potenzialità disponibili. \textbf{Voto 7.5}
	
	\subsection*{Community}
		\paragraph{Unity3D} Community vasta e molto disponibile. Infiniti tutorial online. \textbf{Voto 10}
		\paragraph{Corona SDK} Ottima community e vari tutorial online: il forum aiuta spesso nel caso di problemi particolari. Gli sviluppatori di Corona rispondono sempre e offrono risposte complete. \textbf{Voto 9}
	
	\subsection*{Curva di apprendimento}
		\paragraph{Unity3D} Inizialmente è alta. Inoltre quando le cose diventano complesse è richiesto uno sforzo d'apprendimento notevole. \textbf{Voto 6}
		\paragraph{Corona SDK} Il linguaggio di Lua è un linguaggio semplice soprattutto per chi già utilizza linguaggi di scripting. Corona SDK inizialmente è molto intuitivo. In due settimane di pratica si è operativi ed efficienti. \textbf{Voto 7.5}
	
	\subsection*{Librerie/API offerte}
		\paragraph{Unity3D} Sicuramente presenti per qualsiasi argomento. \textbf{Voto 10}
		\paragraph{Corona SDK} Completo per la gestione degli elementi su schermo e della relativa fisica. Deludente nel campo dei widget. \textbf{Voto 7}
		
	\subsection*{Performance Software prodotto}
		\paragraph{Unity3D} Tutto è lasciato all'esperienza del programmatore. Serve, quindi, molta ottimizzazione nel codice e nei multimedia usati. \textbf{Voto 6}
		\paragraph{Corona SDK} Per quanto visto con il progetto non ho mai avuto cali di frame rate anche nei casi in cui c'erano bug evidenti nel codice. \textbf{Voto 9}
	
	\subsection*{Piattaforme supportate}
		\paragraph{Unity3D} Più di 25 piattaforme: è più semplice elencare quelle non supportate. \textbf{Voto 10}
		\paragraph{Corona SDK} Sistemi Apple iOS, sistemi Android, Amazon Fire, Mac Desktop, Windows Desktop, e anche TV come Apple TV, Fire TV e Android TV. \textbf{Voto 8.5}
	
	\subsection*{Contesti Applicativi}
		\paragraph{Unity3D} Videogame, contesti 3D (per esempio architettura), simulazioni e realtà aumentata. \textbf{Voto 8.5}
		\paragraph{Corona SDK} Solo giochi 2D, sconsigliata per le business app e giochi complessi. \textbf{Voto 6.5}
	
	\subsection*{Licenza}
		\paragraph{Unity3D} Se si ha uno stipendio sotto ai 100.000 \$ annui la licenza è gratuita altrimenti costa 35 \$ al mese, 125 \$ se il fatturato è più alto. \textbf{Voto 7}
		\paragraph{Corona SDK} Totalmente gratuito. \textbf{Voto 10}
		
		
	\begin{table}
		\centering
		\begin{tabular}{lrr}
			\toprule
			\textbf{Caratteristiche} & \textbf{Unity3D} & \textbf{Corona SDK} \\
			\toprule
			\emph{Installazione} & 5 & 9 \\
			
			\emph{Strumenti di sviluppo} & 8.5 & 6.5 \\
			
			\emph{Documentazione} & 10 & 7.5 \\
			
			\emph{Community} & 10 & 9 \\
			
			\emph{Curva di apprendimento} & 6 & 7.5 \\
			
			\emph{Librerie/API offerte} & 10 & 7 \\
			
			\emph{Performance sw prodotto} & 6 & 9 \\
			
			\emph{Piattaforme} & 9 & 8.5 \\
			
			\emph{Contesti applicativi} & 8.5 & 6.5 \\
			
			\emph{Licenza} & 7 & 10 \\
			
			\bottomrule
		\end{tabular}
		\caption{Riassunto comparazione Unity3D e Corona SDK}
	\end{table}
	
	

